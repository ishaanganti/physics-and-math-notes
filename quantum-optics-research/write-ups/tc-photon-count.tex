\documentclass[10pt]{article}
\usepackage[a4paper]{geometry}

\usepackage[utf8]{inputenc}
\usepackage[T1]{fontenc}
\usepackage{textcomp}
\usepackage[dutch]{babel}
\usepackage{amsmath, amssymb}
\usepackage{tcolorbox}
\usepackage{xcolor}
\usepackage{braket}
\definecolor{darkgreen}{rgb}{0.0, 0.5, 0.0}

% figure support
\usepackage{import}
\usepackage{xifthen}
\pdfminorversion=7
\usepackage{pdfpages}
\usepackage{transparent}
\newcommand{\incfig}[1]{%
    \def\svgwidth{\columnwidth}
    \import{./figures/}{#1.pdf_tex}
}

\pdfsuppresswarningpagegroup=1

\usepackage{tikz}
\usepackage{pgfplots}
\usepackage{float}
\pgfplotsset{compat = newest}
\usepgfplotslibrary{colormaps}

%Declarations
\newtheorem{definition}{Definition}
\newtheorem{lemma}{Lemma}
\newtheorem{proof}{Proof}


% Define the 'example' tcolorbox
\newtcolorbox{examplebox}[1][]{
  colback=gray!5,         % Background color
  colframe=darkgreen!75!black, % Frame color
  fonttitle=\bfseries,
  title=Example:~#1,
}


\begin{document}
\title{analytical photon number for the tc model}
\author{Ishaan Ganti}
\date{1 november 2024}
\maketitle

\section{Analytical Treatment of the Tavis-Cummings Model}
The Tavis-Cummings (TC) model is a simple theoretical framework for
describing the interaction between $N$ two level systems (TLS) and an optical
cavity mode. Its Hamiltonian is given by 
\begin{gather}
    H_{TC} = \frac{\omega_m}{2} \sum_{n=1}^{N}  \sigma_z^{n} + \omega_c 
    a^{\dag }a + 
    \frac{g}{\sqrt{N}} \sum_{n=1}^{N} (\sigma_{+}a + \sigma_{-}a^{\dag }) 
\end{gather}
where $\omega_m$ is the TLS transition frequency, $\omega_c$ is the cavity frequency, 
and $\frac{g}{\sqrt{N} }$ gives the collective coupling strength. Clearly, this model
conserves excitations. Thus, we can solve the model analytically within an excitation subspace.
Consider the second excitation manifold, whose symmetrized bright basis states are given by 
\begin{gather}
    \ket{s_0, 2}, \, \ket{s_1, 1}, \,  \ket{s_2, 0} 
\end{gather}

\subsection{Subspace Hamiltonian}

We proceed by deriving Hamiltonian matrix in this subspace. The diagonal elements must be given
by $2\omega_c$, $\omega_c + \omega_m$, and $2\omega_m$, respectively. Additionally, since the
first and third basis states aren't coupled directly, we must have that the upper right and lower
left entries are 0. Solving for the coupling terms benefits from working
in the angular momentum basis, with resulting Hamiltonian
\begin{gather}
    H_{TC} = \frac{\omega_m}{2} J_z + 
    \omega_c a^{\dag }a + 
    \frac{g}{\sqrt{N}} \left( J_{+}a + J_{-}a^{\dag } \right) 
\end{gather}
We calculate the following action
\begin{gather}
    J_{+}a \ket{s_1, 1} = J_{+}\ket{j = N / 2, m = -j + 1} 
    \otimes a\ket{1}
\end{gather}
The coefficient attached to the TLS ensemble term is given by 
\begin{gather*}
    \sqrt{j(j+1) - m(m+1)} = 
    \sqrt{j(j+1) - (1-j)(2-j)} = \sqrt{4j - 2} = \sqrt{2N - 2}   
\end{gather*}
So eq. (4) becomes
\begin{gather}
    \sqrt{2(N-1)} \ket{s_2, 0}  
\end{gather}
Similarly
\begin{gather}
    J_{-}a^{\dag } \ket{s_1, 1} =   
    \sqrt{2N} \ket{s_0, 2} 
\end{gather}
Using these results, we can calculate the matrix elements at (0, 1) and
(1, 2) as follows
\begin{gather}
    \braket{s_2, 0|H_{INT} | s_1, 1} = 
    \frac{g}{\sqrt{N}}\sqrt{2(N-1)} 
    \braket{s_2, 0|s_2, 0} = g\sqrt{\frac{2(N-1)}{N}} \\ 
    \braket{s_0, 2| H_{INT}| s_1, 1} = 
    \frac{g}{\sqrt{N}} \sqrt{2N}
    \braket{s_0, 2| s_0, 2} = g\sqrt{2}  
\end{gather}
Put $\Omega_k = g\sqrt{2K / N}$; then, our Hamiltonian matrix becomes
\begin{gather}
    \begin{pmatrix} 2 \omega_c & \Omega_{N} & 0 \\
        \Omega_{N} & \omega_c + \omega_m & \Omega_{N-1} \\
    0 & \Omega_{N-1} & 2\omega_m \end{pmatrix} 
\end{gather}

\subsection{Resonant Solution}
We limit our focus to a small number of TLS--specifically, we do not 
take the thermodynamic limit in which $N \gg 1$, which would've allowed
for the approximation $\Omega_{N-1} \approx \Omega_{N}$. This system 
can still be solved exactly, but doing so is \textit{extremely} bashy and
realistically requires the assistance of a symbolic calculator. However, 
given resonance $(\omega_m = \omega_c)$, the computation becomes much 
more managable. Additionally, the resonance dynamics are of significant
interest. 

\subsubsection{Eigenenergies \& Eigenstates}
We have
\begin{gather*} 
    0 = \text{det} \left (\begin{pmatrix} 2 \omega & \Omega_{N} & 0 \\
        \Omega_{N} & 2\omega & \Omega_{N-1} \\
    0 & \Omega_{N-1} & 2\omega \end{pmatrix} - \lambda I\right ) = 
            \begin{vmatrix} 2 \omega - \lambda & \Omega_{N} & 0 \\
        \Omega_{N} & 2\omega - \lambda & \Omega_{N-1} \\
    0 & \Omega_{N-1} & 2\omega - \lambda\end{vmatrix} \\ = 
    (2\omega - \lambda)((2\omega - \lambda)^2 - \Omega_{N-1}^2) - 
    \Omega_{N}(\Omega_{N}(2\omega -\lambda)) \\ = 
    (2\omega - \lambda)[(2\omega -\lambda)^2 - \Omega_N^2 - 
    \Omega_{N-1}^2]
\end{gather*}
This leads to three polariton solutions
\begin{gather}
    \lambda = 2\omega , \quad 
    \lambda_{\pm} = 2\omega \pm \sqrt{\Omega_N^2 + \Omega_{N-1}^2} 
\end{gather}
Notably, the upper and lower polariton energies average to exactly the middle
polariton energy. One may solve for the eigenstates analytically as well; 
the results are (in ascending order) 
\begin{gather}
    \begin{pmatrix}
        \sqrt{\frac{N}{N-1}} \\
        -\sqrt{\frac{2N-1}{N-1}} \\
        1
    \end{pmatrix},
    \quad
    \begin{pmatrix}
        -\sqrt{\frac{N-1}{N }} \\
        0 \\
        1
    \end{pmatrix},
    \quad
    \begin{pmatrix}
        \sqrt{\frac{N}{N-1}} \\
        \sqrt{\frac{2N-1}{N-1}} \\
        1
    \end{pmatrix}
\end{gather}

\subsubsection{Cavity Photon Expectation}
we aim to calculate the dynamics of the cavity photon expectation value. 
it is clear that in this basis, the cavity number operator is given by
\begin{gather}
    \hat{n} = 
    \begin{pmatrix} 2 & 0 & 0 \\ 0 & 1 & 0 \\ 0 & 0 & 0 \end{pmatrix} 
\end{gather}
additionally, any state in this basis can be written like
\begin{gather}
    \psi(t) = c_0 e^{-i\lambda_{-} t} \ket{\lambda_{-}} + c_1
    e^{-i\lambda t} \ket{\lambda} + c_2 e^{-i\lambda_{+} t} \ket{\lambda_{+}} 
\end{gather}
we need the action of the number operator acting on each eigenstate. 
for simplicity, we put $\alpha = \sqrt{N / (N-1)}$, $\beta = \sqrt{(2N-1) / (N-1)}$
as well as normalization constants $\gamma_1$ and $\gamma_2$. 
\begin{gather*}
    \begin{pmatrix} 2 & 0 & 0 \\ 0 & 1 & 0 \\ 0 & 0 & 0 \end{pmatrix}
    \gamma_1 \begin{pmatrix} \alpha \\ -\beta \\ 1 \end{pmatrix} = \gamma_1
    \begin{pmatrix} 2\alpha \\ - \beta \\ 0 \end{pmatrix} \\
    \begin{pmatrix} 2 & 0 & 0 \\ 0 & 1 & 0 \\ 0 & 0 & 0 \end{pmatrix}
    \gamma_2 \begin{pmatrix} - 1 / \alpha \\ 0 \\ 1 \end{pmatrix} = \gamma_2
    \begin{pmatrix} -2 / \alpha \\ 0 \\ 0 \end{pmatrix} \\
        \begin{pmatrix} 2 & 0 & 0 \\ 0 & 1 & 0 \\ 0 & 0 & 0 \end{pmatrix}
    \gamma_1 \begin{pmatrix} \alpha \\ \beta \\ 1 \end{pmatrix} = \gamma_1
    \begin{pmatrix} 2\alpha \\ \beta \\ 0 \end{pmatrix} \\
\end{gather*}
consider an initial state with all excitations in the tls ensemble i.e. 
$\ket{s_2, 0}$. the projection of this initial state onto the eigenbasis
yields projection coefficients $c_0 = \gamma_1$, $c_1 = \gamma_2$, and
$c_2 = \gamma_1$. with this, we may expand $\braket{\psi(t)|\hat{n} | \psi(t)}$
to solve for $\left < \hat{n} \right >(t)$. The calculation
is also quite bashy; hence, omitted, but it yields the following:
\begin{gather}
    \left < n \right > (t) = 
    -\frac{2(N-1)(1-4N + \cos (\Delta \lambda \, t)) 
    }{(2N-1)^2}
    \sin^2 \left(\frac{\Delta \lambda \,  t}{2}\right)
\end{gather}
Where we've put $\Delta \lambda = \sqrt{\Omega_N^2 + \Omega_{N-1}^2}$. We
can confirm this result numerically; consider $N = 2$ with coupling
$g = 0.07$; this gives $\Delta \lambda \approx 0.12$. The resulting plot is

\begin{center}    
    \begin{figure} [H]
        \includegraphics[width=\linewidth, height=8.1cm]{analytical-check.png}
        \caption{Analytical expression and numerical results plotted on
        the same graph.}
    \end{figure}
\end{center}
Essentially an exact agreement. 

\end{document}

