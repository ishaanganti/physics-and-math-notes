
\documentclass[a4paper,10pt]{article}
\usepackage[utf8]{inputenc}
\usepackage[a4paper,margin=3.5cm]{geometry} % Sets the page geometry
\setlength{\parindent}{0pt} % Eliminates indentation at the beginning of paragraphs
\setlength{\parskip}{0.5em} % Adds vertical space between paragraphs

\usepackage{url}
\usepackage{dirtytalk}
\usepackage{graphicx} % Package for \includegraphics
\usepackage{wrapfig} % Figure wrapping
\usepackage[T1]{fontenc} % Output font encoding for international characters
\setlength{\parskip}{1em} % Set space between paragraphs
\usepackage{amssymb}
\usepackage{amsmath}
\usepackage{tcolorbox} % For adding boxes around theorems and proofs
\tcbuselibrary{theorems, breakable} % Allows tcolorboxes to break across pages and manage theorem environments
\usepackage{mathtools}
\usepackage{amsthm} % Package for theorem environments

% Define the main theorem environments with proper counters
\theoremstyle{plain}
\newtheorem{theorem}{Theorem}[section]
\newtheorem{proposition}{Proposition}[section]
\newtheorem{definition}{Definition}[section]
\newtheorem{corollary}[theorem]{Corollary}
\newtheorem{lemma}[theorem]{Lemma}

% Define theorem and proof environments with tcolorbox, linking to the standard theorem counters
\newtcbtheorem[use counter=theorem]{mytheorem}{Theorem}{
    breakable,
    colback=white,
    colframe=black,
    boxrule=0.5pt,
    sharp corners,
    left=0mm,
    right=0mm,
    top=0mm,
    bottom=0mm
}{thm}
\newtcbtheorem[ use counter=proposition]{myproposition}{Theorem}{
    breakable,
    colback=white,
    colframe=blue,
    boxrule=0.5pt,
    sharp corners,
    left=0mm,
    right=0mm,
    top=0mm,
    bottom=0mm
}{thm}

\newtcolorbox{myproof}[1][]{
    breakable,
    colback=white,
    colframe=gray,  
    boxrule=0.5pt,
    sharp corners,
    left=0mm,
    right=0mm,
    top=0mm,
    bottom=0mm,
    title=Proof,
    fonttitle=\bfseries,
    #1,
    parbox=false, % Ensures that text justification is handled properly
    before upper={\parindent0pt \parskip0.5em}, % Sets no indentation and a space between paragraphs
}

\begin{document}
\title{Chapter 1 Reading Notes}
\author{Ishaan Ganti \\ Principles of Mathematical Analysis, Rudin}
\date{\today}
\maketitle
\section{Introduction}
We start by discussing an `incompleteness' of the rationals that 
motivates the study of the reals. Put bluntly, there exists no
such $p \in \mathbb{Q}$ that satisfies $p^2 = 2$. The proof of this 
is pretty straightforward and is widely found, hence omitted. Instead,
I will prove a related result. 

\begin{mytheorem}{}{}
    Let $A$ be the set of all $a \in Q$ such that $a^2 < 2$. Similarly, let $B$ 
    be the set of all $b \in Q$ such that $b^2 > 2$. Then $A$ has no largest
    element and $B$ has no smallest element. 
\end{mytheorem}

\begin{myproof}
    For any $p \in Q$, consider the value given by 
    \[
        \gamma = p - \frac{p^2 - 2}{p + 2}
    \]
    If $p \in A$, then either $p^2 - 2 < 0$ and $p + 2 > 0$ or $p^2 - 2 > 0$ and $p + 2 < 0$, so
    $\gamma > p$. Then, we calculate $\gamma^2-2$
    \begin{gather*}
        \gamma = \frac{p^2 + 2p - p^2+2}{p+2} = \frac{2p + 2}{p + 2} \\
        \gamma^2 - 2 = 
        \frac{4p^2 + 8p + 4 - 2p^2 - 8p - 8}{(p+2)^2} = \frac{2p^2 - 4}{(p+2)^2} = \frac{2(p^2 - 2)}{(p+2)^2}  
    \end{gather*}
    But then $p^2 - 2$ in the numerator is negative, so $\gamma^2 -2 <0$ and $\gamma \in A$. Practically
    the same logic applies for showing that $B$ has no smallest element. 
\end{myproof}

The bit of this proof that was tricky for me was defining the expression that 
explicitly shows that $A$ has no largest element and $B$ has no smallest element. The proof
statement itself, however, felt pretty intuitive. 

\begin{mytheorem}{Least upper bound property implies greatest lower bound}{}
    Suppose \(S\) is an ordered set with the least upper-bound property, 
    \(B \subset S\), \(B\) is not empty, and \(B\) is bounded below. Let \(L\) 
    be the set of all lower bounds of \(B\). Then 
    \[
        \alpha = \sup L
    \]
    exists in \(S\) and \(\alpha = \inf B\). 
\end{mytheorem}

\begin{myproof}
    First, we show that \(L\) is bounded above. Since for all \(b \in B\), 
    \(l \in L\), we know \(l \le b\), all \(b\) serve as upper bounds for
    \(L\). Then, since \(S\) possesses the least upper-bound property, there
    exists some \(\alpha = \sup L\) in \(S\). This proves the first part of 
    the theorem.

    Now, consider some other \(\gamma\) in \(S\). Assume \(\gamma > \alpha\). Then, 
    since \(\alpha\) is the supremum of \(L\), \(\gamma \notin L\). By
    the definition of \(L\), \(\gamma\) is not a lower bound of \(B\). However, 
    if \(\gamma < \alpha\), then it cannot be the infimum of \(B\) as 
    \(\alpha\) bounds \(B\) from below as \(\alpha \in L\) and \(\alpha > \gamma\).
    Thus, \(\alpha\) must be the infimum of \(B\). 
\end{myproof}
\textbf{Note:} paying attention to definitions is important. \(B\) being bounded below
as a subset of \(S\) implies that all values that bound \(B\) below \textit{are within}
\(S\) by the definition of bounded below. So, we must have \(L \subset S\). 

\section{Fields}
The text goes on to discuss fields in some length. I will skip over the introductory propositions
as I have experience in algebra, but I will prove one set of propositions just as a sort of
checkpoint. 

\begin{myproposition}{}{}
    The following statements are true in every ordered field. 
    \begin{enumerate}
        \item If $x > 0$ then $-x < 0$ and vice versa. 
        \item If $x > 0 $ and $y < z$ then  $xy < xz$.
        \item If $x < 0$ and $y < z$ then $xy > xz$.
        \item If  $x \neq 0$ then  $x^2 > 0$. In particular, $1 > 0$. 
        \item If $0 < x < y$ then $0 < \frac{1}{y} < \frac{1}{x}$. 
    \end{enumerate}
\end{myproposition}

\begin{myproof}
    \begin{enumerate}
        \item Since $x > 0$, $-x + x > -x$, but this is just $0 > -x$ or $-x < 0$.
        \item $xz = x(z-y) + xy > 0 + xy = xy$. 
        \item $xy = x(y-z) + xz > 0 + xz > xz$. 
    \end{enumerate}
\end{myproof}

I got too lazy to do the rest; they follow from the definitions and some of the earlier 
propositions. 

\section{The Real Field}

\end{document}
